\documentclass[a4,12pt]{article}
\usepackage[utf8]{inputenc}
\usepackage[T2A]{fontenc}
\usepackage[english, russian]{babel}


\usepackage{csquotes}
\usepackage{epigraph}
\usepackage{tcolorbox}
% Определяем окружение для Формата ввода
\newtcolorbox{inputformat}[1][]{colback=blue!5!white, colframe=blue!65!black,
    fonttitle=\bfseries, title=Формат ввода, #1}

% Определяем окружение для Формата вывода
\newtcolorbox{outputformat}[1][]{colback=green!5!white, colframe=green!65!black,
    fonttitle=\bfseries, title=Формат вывода, #1}

% Определяем окружение для комментария
\newtcolorbox{exercisecomment}[1][]{colback=yellow!5!white, colframe=yellow!80!black, fonttitle=\bfseries, title=Комментарий, #1}

% Определяем окружение для замечания
\newtcolorbox{exercisenote}[1][]{colback=red!5!white, colframe=red!80!black, fonttitle=\bfseries, title=Замечание, #1}


\usepackage{geometry}
%\geometry{papersize={20.3 cm,25.4 cm}}
\geometry{left=2cm}
\geometry{right=2cm}
\geometry{top=2cm}
\geometry{bottom=2cm}


\usepackage{amsmath, amsfonts, amsthm, amssymb, amsopn, amscd}
\usepackage{enumerate}
\usepackage{enumitem}
\usepackage[mathscr]{eucal}

\usepackage{hyperref}
\hypersetup{unicode=true,final=true,colorlinks=true}

\theoremstyle{remark}
%\newtheorem{exercise}{Упражнение}
\newtheorem{exercise}{\textbf{Упражнение}}[section]
\renewcommand{\theexercise}{\textbf{\arabic{exercise}}}
%\renewcommand{\theexercise}{\textbf{\#\arabic{exercise}}}



\title{Листок 04. k-NN (k Nearest Neighbors)}

\author{А.Н. Лата}

\begin{document}

\section*{\centering Листок 04. k-NN (k Nearest Neighbors)}

\begin{exercise}
Для набора данных \texttt{sleep75} рассмотрим задачу прогнозирования
для переменных
\begin{center}
	\begin{tabular}{|c|c|}\hline
		зависимая/target & объясняющая/предикторы/features \\ \hline
		sleep & totwrk, age, south, male \\ \hline
	\end{tabular}
\end{center}
\begin{enumerate}
	\item подгоните на исходном датасете модель k-NN с параметрами
	\begin{center}
		\begin{tabular}{|l|l|l|}\hline
		№ & \(k\) & веса \\ \hline
		1 & 5 & uniform \\
		2 & 5 & distance \\
		3 & 10 & uniform \\
		4 & 10 & distance \\ \hline
		\end{tabular}
	\end{center}
	\item Рассмотрим трёх людей с характеристиками
	\begin{center}
		\begin{tabular}{|l||l|l|l|l|}\hline
			index & totwrk & age & south & male \\ \hline\hline
			0 & 2160 & 32 & 1 & 0 \\
			1 & 1720 & 24 & 0 & 1 \\
			2 & 2390 & 44 & 0 & 1 \\ \hline
		\end{tabular}
	\end{center}
	вычислите прогноз \textbf{sleep} по каждой модели
\end{enumerate}
\end{exercise}

\begin{exercise}
Для набора данных \texttt{sleep75} рассмотрим задачу прогнозирования
для переменных
\begin{center}
	\begin{tabular}{|c|c|}\hline
		зависимая/target & объясняющая/предикторы/features \\ \hline
		sleep & totwrk, age, south, male, smsa, yngkid, marr \\ \hline
	\end{tabular}
\end{center}
\begin{enumerate}
	\item подгоните на исходном датасете модель k-NN с параметрами
	\begin{center}
		\begin{tabular}{|l|l|l|}\hline
		№ & \(k\) & веса \\ \hline
		1 & 5 & uniform \\
		2 & 5 & distance \\
		3 & 10 & uniform \\
		4 & 10 & distance \\ \hline
		\end{tabular}
	\end{center}
	\item Рассмотрим трёх людей с характеристиками
	\begin{center}
		\begin{tabular}{|l||l|l|l|l|l|l|l|}\hline
			index & totwrk & age & south & male & smsa & yngkid & marr \\ \hline\hline
			0 & 2150 & 37 & 0 & 1 & 1 & 0 & 1 \\
			1 & 1950 & 28 & 1 & 1 & 0 & 1 & 0 \\
			2 & 2240 & 26 & 0 & 0 & 1 & 0 & 0 \\ \hline
		\end{tabular}
	\end{center}
	вычислите прогноз \textbf{sleep} по каждой модели
\end{enumerate}
\end{exercise}

\begin{exercise}
Для набора данных \texttt{wage2} рассмотрим задачу прогнозирования
для переменных
\begin{center}
	\begin{tabular}{|c|c|}\hline
		зависимая/target & объясняющая/предикторы/features \\ \hline
		wage & age, IQ, south, married, urban \\ \hline
	\end{tabular}
\end{center}
\begin{enumerate}
	\item подгоните на исходном датасете модель k-NN с параметрами
	\begin{center}
		\begin{tabular}{|l|l|l|}\hline
		№ & \(k\) & веса \\ \hline
		1 & 5 & uniform \\
		2 & 5 & distance \\
		3 & 10 & uniform \\
		4 & 10 & distance \\ \hline
		\end{tabular}
	\end{center}
	\item Рассмотрим трёх людей с характеристиками
	\begin{center}
		\begin{tabular}{|l||l|l|l|l|l|}\hline
			index & age & IQ & south & married & urban \\ \hline\hline
			0 & 36 & 105 & 1 & 1 & 1 \\
			1 & 29 & 123 & 0 & 1 & 0 \\
			2 & 25 & 112 & 1 & 0 & 1 \\ \hline
		\end{tabular}
	\end{center}
	вычислите прогноз \textbf{wage} по каждой модели
\end{enumerate}
\end{exercise}

\begin{exercise}
Для набора данных \texttt{wage2} рассмотрим задачу прогнозирования
для переменных
\begin{center}
	\begin{tabular}{|c|c|}\hline
		зависимая/target & объясняющая/предикторы/features \\ \hline
		log(wage) & age, IQ, south, married, urban \\ \hline
	\end{tabular}
\end{center}
\begin{enumerate}
	\item подгоните на исходном датасете модель k-NN с параметрами
	\begin{center}
		\begin{tabular}{|l|l|l|}\hline
		№ & \(k\) & веса \\ \hline
		1 & 5 & uniform \\
		2 & 5 & distance \\
		3 & 10 & uniform \\
		4 & 10 & distance \\ \hline
		\end{tabular}
	\end{center}
	\item Рассмотрим трёх людей с характеристиками
	\begin{center}
		\begin{tabular}{|l||l|l|l|l|l|}\hline
			index & age & IQ & south & married & urban \\ \hline\hline
			0 & 36 & 105 & 1 & 1 & 1 \\
			1 & 29 & 123 & 0 & 1 & 0 \\
			2 & 25 & 112 & 1 & 0 & 1 \\ \hline
		\end{tabular}
	\end{center}
	вычислите прогноз \textbf{wage} по каждой модели
\end{enumerate}
\end{exercise}

\begin{exercise}
Для набора данных \texttt{wage1} рассмотрим задачу прогнозирования
для переменных
\begin{center}
	\begin{tabular}{|c|c|}\hline
		зависимая/target & объясняющая/предикторы/features \\ \hline
		wage & exper, female, married, smsa \\ \hline
	\end{tabular}
\end{center}
\begin{enumerate}
	\item подгоните на исходном датасете модель k-NN с параметрами
	\begin{center}
		\begin{tabular}{|l|l|l|}\hline
		№ & \(k\) & веса \\ \hline
		1 & 5 & uniform \\
		2 & 5 & distance \\
		3 & 10 & uniform \\
		4 & 10 & distance \\ \hline
		\end{tabular}
	\end{center}
	\item Рассмотрим трёх людей с характеристиками
	\begin{center}
		\begin{tabular}{|l||l|l|l|l|}\hline
			index & exper & female & married & smsa \\ \hline\hline
			0 & 5 & 1 & 1 & 1  \\
			1 & 26 & 0 & 0 & 1 \\
			2 & 38 & 1 & 1 & 0 \\ \hline
		\end{tabular}
	\end{center}
	вычислите прогноз \textbf{wage} по каждой модели
\end{enumerate}
\end{exercise}

\begin{exercise}
Для набора данных \texttt{wage1} рассмотрим задачу прогнозирования
для переменных
\begin{center}
	\begin{tabular}{|c|c|}\hline
		зависимая/target & объясняющая/предикторы/features \\ \hline
		log(wage) & exper, female, married, smsa \\ \hline
	\end{tabular}
\end{center}
\begin{enumerate}
	\item подгоните на исходном датасете модель k-NN с параметрами
	\begin{center}
		\begin{tabular}{|l|l|l|}\hline
		№ & \(k\) & веса \\ \hline
		1 & 5 & uniform \\
		2 & 5 & distance \\
		3 & 10 & uniform \\
		4 & 10 & distance \\ \hline
		\end{tabular}
	\end{center}
	\item Рассмотрим трёх людей с характеристиками
	\begin{center}
		\begin{tabular}{|l||l|l|l|l|}\hline
			index & exper & female & married & smsa \\ \hline\hline
			0 & 5 & 1 & 1 & 1  \\
			1 & 26 & 0 & 0 & 1 \\
			2 & 38 & 1 & 1 & 0 \\ \hline
		\end{tabular}
	\end{center}
	вычислите прогноз \textbf{wage} по каждой модели
\end{enumerate}
\end{exercise}

\begin{exercise}
Для набора данных \texttt{Labour} рассмотрим задачу прогнозирования
для переменных
\begin{center}
	\begin{tabular}{|c|c|}\hline
		зависимая/target & объясняющая/предикторы/features \\ \hline
		output & capital, labour \\ \hline
	\end{tabular}
\end{center}
\begin{enumerate}
	\item подгоните на исходном датасете модель k-NN с параметрами
	\begin{center}
		\begin{tabular}{|l|l|l|}\hline
		№ & \(k\) & веса \\ \hline
		1 & 5 & uniform \\
		2 & 5 & distance \\
		3 & 10 & uniform \\
		4 & 10 & distance \\ \hline
		\end{tabular}
	\end{center}
	\item Рассмотрим трёх людей с характеристиками
	\begin{center}
		\begin{tabular}{|l||l||l|l|}\hline
			index & capital & labour \\ \hline\hline
			0 & 2.970 & 85 \\
			1 & 10.450 & 60  \\
			2 & 3.850 & 105 \\ \hline
		\end{tabular}
	\end{center}
	вычислите прогноз \textbf{output} по каждой модели
\end{enumerate}
\end{exercise}

\begin{exercise}
Для набора данных \texttt{Labour} рассмотрим задачу прогнозирования
для переменных
\begin{center}
	\begin{tabular}{|c|c|}\hline
		зависимая/target & объясняющая/предикторы/features \\ \hline
		log(output) & log(capital), log(labour) \\ \hline
	\end{tabular}
\end{center}
\begin{enumerate}
	\item подгоните на исходном датасете модель k-NN с параметрами
	\begin{center}
		\begin{tabular}{|l|l|l|}\hline
		№ & \(k\) & веса \\ \hline
		1 & 5 & uniform \\
		2 & 5 & distance \\
		3 & 10 & uniform \\
		4 & 10 & distance \\ \hline
		\end{tabular}
	\end{center}
	\item Рассмотрим трёх людей с характеристиками
	\begin{center}
		\begin{tabular}{|l||l||l|l|}\hline
			index & capital & labour \\ \hline\hline
			0 & 2.970 & 85 \\
			1 & 10.450 & 60  \\
			2 & 3.850 & 105 \\ \hline
		\end{tabular}
	\end{center}
	вычислите прогноз \textbf{output} по каждой модели
\end{enumerate}
\end{exercise}

\begin{exercise}
Для набора данных \texttt{Labour} рассмотрим задачу прогнозирования
для переменных
\begin{center}
	\begin{tabular}{|c|c|}\hline
		зависимая/target & объясняющая/предикторы/features \\ \hline
		output & capital, labour, wage \\ \hline
	\end{tabular}
\end{center}
\begin{enumerate}
	\item подгоните на исходном датасете модель k-NN с параметрами
	\begin{center}
		\begin{tabular}{|l|l|l|}\hline
		№ & \(k\) & веса \\ \hline
		1 & 5 & uniform \\
		2 & 5 & distance \\
		3 & 10 & uniform \\
		4 & 10 & distance \\ \hline
		\end{tabular}
	\end{center}
	\item Рассмотрим трёх людей с характеристиками
	\begin{center}
		\begin{tabular}{|l||l|l|l|}\hline
			index & capital & labour & wage \\ \hline\hline
			0 & 2.970 & 85 & 36.98\\
			1 & 10.450 & 60 & 33.82  \\
			2 & 3.850 & 105 & 40.23\\ \hline
		\end{tabular}
	\end{center}
	вычислите прогноз \textbf{output} по каждой модели
\end{enumerate}
\end{exercise}

\begin{exercise}
Для набора данных \texttt{Labour} рассмотрим задачу прогнозирования
для переменных
\begin{center}
	\begin{tabular}{|c|c|}\hline
		зависимая/target & объясняющая/предикторы/features \\ \hline
		log(output) & log(capital), log(labour), log(wage) \\ \hline
	\end{tabular}
\end{center}
\begin{enumerate}
	\item подгоните на исходном датасете модель k-NN с параметрами
	\begin{center}
		\begin{tabular}{|l|l|l|}\hline
		№ & \(k\) & веса \\ \hline
		1 & 5 & uniform \\
		2 & 5 & distance \\
		3 & 10 & uniform \\
		4 & 10 & distance \\ \hline
		\end{tabular}
	\end{center}
	\item Рассмотрим трёх людей с характеристиками
	\begin{center}
		\begin{tabular}{|l||l|l|l|}\hline
			index & capital & labour & wage \\ \hline\hline
			0 & 2.970 & 85 & 36.98\\
			1 & 10.450 & 60 & 33.82  \\
			2 & 3.850 & 105 & 40.23\\ \hline
		\end{tabular}
	\end{center}
	вычислите прогноз \textbf{output} по каждой модели
\end{enumerate}
\end{exercise}

\end{document}