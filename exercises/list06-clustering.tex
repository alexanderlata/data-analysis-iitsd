\documentclass[a4,12pt]{article}
\usepackage[utf8]{inputenc}
\usepackage[T2A]{fontenc}
\usepackage[english, russian]{babel}


\usepackage{csquotes}
\usepackage{epigraph}
\usepackage{tcolorbox}
% Определяем окружение для Формата ввода
\newtcolorbox{inputformat}[1][]{colback=blue!5!white, colframe=blue!65!black,
    fonttitle=\bfseries, title=Формат ввода, #1}

% Определяем окружение для Формата вывода
\newtcolorbox{outputformat}[1][]{colback=green!5!white, colframe=green!65!black,
    fonttitle=\bfseries, title=Формат вывода, #1}

% Определяем окружение для комментария
\newtcolorbox{exercisecomment}[1][]{colback=yellow!5!white, colframe=yellow!80!black, fonttitle=\bfseries, title=Комментарий, #1}

% Определяем окружение для замечания
\newtcolorbox{exercisenote}[1][]{colback=red!5!white, colframe=red!80!black, fonttitle=\bfseries, title=Замечание, #1}


\usepackage{geometry}
%\geometry{papersize={20.3 cm,25.4 cm}}
\geometry{left=2cm}
\geometry{right=2cm}
\geometry{top=2cm}
\geometry{bottom=2cm}


\usepackage{amsmath, amsfonts, amsthm, amssymb, amsopn, amscd}
\usepackage{enumerate}
\usepackage{enumitem}
\usepackage[mathscr]{eucal}

\usepackage{hyperref}
\hypersetup{unicode=true,final=true,colorlinks=true}

\theoremstyle{remark}
%\newtheorem{exercise}{Упражнение}
\newtheorem{exercise}{\textbf{Упражнение}}[section]
\renewcommand{\theexercise}{\textbf{\arabic{exercise}}}
%\renewcommand{\theexercise}{\textbf{\#\arabic{exercise}}}



\title{Листок 06. Задача кластеризации}

\author{А.Н. Лата}

\begin{document}

\section*{\centering Листок 06. Задача кластеризации}


\textbf{Замечание} \textit{обязательно проводим предварительную обработку данных:
удаление пропущенных значений, нормировку, преобразование категориальных признаков}

\begin{exercise}
Для набора данных \texttt{countries} проведите разбиение на кластеры следующими
методам:
\begin{center}
	\begin{tabular}{c|l}
		Число кластеров & Метод \\ \hline
		3 & k-средних \\
		4 & k-средних \\
		5 & k-средних \\
		3 & иерархическая \\
		4 & иерархическая \\
		5 & иерархическая \\ \hline
	\end{tabular}
\end{center}
Визуализируйте разбиение на кластеры на диаграмме рассеяния в переменных датасета
\end{exercise}

\begin{exercise}
Для набора данных \texttt{countries} найдите <<оптимальное>> число кластеров
для метода
\begin{enumerate}
	\item k-средних
	\item иерархической кластеризации
\end{enumerate}
относительно метрик: Silhouette, Calinski-Harabasz, Davies-Bouldin
\end{exercise}

\begin{exercise}
Из набора данных \texttt{countries} возьмите переменные 
\texttt{sleep, totwrk, age, educ} и проведите разбиение на кластеры следующими
методам:
\begin{center}
	\begin{tabular}{c|l}
		Число кластеров & Метод \\ \hline
		3 & k-средних \\
		4 & k-средних \\
		5 & k-средних \\
		3 & иерархическая \\
		4 & иерархическая \\
		5 & иерархическая \\ \hline
	\end{tabular}
\end{center}
Визуализируйте разбиение на кластеры на диаграмме рассеяния в переменных датасета
\end{exercise}



\begin{exercise}
Из набора данных \texttt{countries} возьмите переменные 
\texttt{sleep, totwrk, age, educ} и найдите <<оптимальное>> число кластеров
для метода
\begin{enumerate}
	\item k-средних
	\item иерархической кластеризации
\end{enumerate}
относительно метрик: Silhouette, Calinski-Harabasz, Davies-Bouldin
\end{exercise}

\begin{exercise}
Для набора данных \texttt{Labour} проведите разбиение на кластеры следующими
методам:
\begin{center}
	\begin{tabular}{c|l}
		Число кластеров & Метод \\ \hline
		3 & k-средних \\
		4 & k-средних \\
		5 & k-средних \\
		3 & иерархическая \\
		4 & иерархическая \\
		5 & иерархическая \\ \hline
	\end{tabular}
\end{center}
Визуализируйте разбиение на кластеры на диаграмме рассеяния в переменных датасета
\end{exercise}
	
\begin{exercise}
Для набора данных \texttt{Labour} найдите <<оптимальное>> число кластеров
для метода
\begin{enumerate}
	\item k-средних
	\item иерархической кластеризации
\end{enumerate}
относительно метрик: Silhouette, Calinski-Harabasz, Davies-Bouldin
\end{exercise}

\end{document}