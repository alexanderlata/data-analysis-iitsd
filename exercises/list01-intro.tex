\documentclass[a4,12pt]{article}
\usepackage[utf8]{inputenc}
\usepackage[T2A]{fontenc}
\usepackage[english, russian]{babel}


\usepackage{csquotes}
\usepackage{epigraph}
\usepackage{tcolorbox}
% Определяем окружение для Формата ввода
\newtcolorbox{inputformat}[1][]{colback=blue!5!white, colframe=blue!65!black,
    fonttitle=\bfseries, title=Формат ввода, #1}

% Определяем окружение для Формата вывода
\newtcolorbox{outputformat}[1][]{colback=green!5!white, colframe=green!65!black,
    fonttitle=\bfseries, title=Формат вывода, #1}

% Определяем окружение для комментария
\newtcolorbox{exercisecomment}[1][]{colback=yellow!5!white, colframe=yellow!80!black, fonttitle=\bfseries, title=Комментарий, #1}

% Определяем окружение для замечания
\newtcolorbox{exercisenote}[1][]{colback=red!5!white, colframe=red!80!black, fonttitle=\bfseries, title=Замечание, #1}


\usepackage{geometry}
%\geometry{papersize={20.3 cm,25.4 cm}}
\geometry{left=2cm}
\geometry{right=2cm}
\geometry{top=2cm}
\geometry{bottom=2cm}


\usepackage{amsmath, amsfonts, amsthm, amssymb, amsopn, amscd}
\usepackage{enumerate}
\usepackage{enumitem}
\usepackage[mathscr]{eucal}

\usepackage{hyperref}
\hypersetup{unicode=true,final=true,colorlinks=true}

\theoremstyle{remark}
%\newtheorem{exercise}{Упражнение}
\newtheorem{exercise}{\textbf{Упражнение}}[section]
\renewcommand{\theexercise}{\textbf{\arabic{exercise}}}
%\renewcommand{\theexercise}{\textbf{\#\arabic{exercise}}}


\title{Листок 01. Описательные статистики}

\author{А.Н. Лата}

\begin{document}

\section*{\centering Листок 01. Описательные статистики}
\subsection*{Библиотека Pandas}
\begin{exercise}
Загрузите датасет \texttt{sleep75}.
\begin{enumerate}
	\item вычислите размер датасета (число наблюдений \& число переменных)
	\item Заполните следующую таблицу со значениями переменных
	\begin{center}
		\begin{tabular}{|c|c|c|c|c|} \hline
			index & sleep & totwrk & age & male\\ \hline\hline
			0 & & & & \\ \hline
			5 & & & & \\ \hline
			100 & & & & \\ \hline
			700 & & & & \\ \hline
		\end{tabular}
	\end{center}
	\item Вычислите корреляционную матрицу для следующих переменных: sleep, totwrk, age 
	\item Заполните следующую таблицу
	\begin{center}
		\begin{tabular}{|c|c|c|c|c|} \hline
			Desc.Stat & sleep & totwrk & age & hrwage\\ \hline\hline
			max & & & & \\ \hline
			min & & & & \\ \hline
			mean & & & & \\ \hline
			median & & & & \\ \hline
			st.dev & & & & \\ \hline
			var (unbiased) & & & & \\ \hline
			var (biased) & & & & \\ \hline
			1st quartile & & & & \\ \hline
			3rd quartile & & & & \\ \hline
		\end{tabular}
	\end{center}
	Замечание: 1st/3rd квантили -- 25\%/75\% квантили соответственно.
	\item Сколько наблюдения соответствуют следующим условиям
	\begin{enumerate}
		\item sleep>3000
		\item totwrk<2000
		\item age>40
		\item age<30
	\end{enumerate}
	\item Сколько наблюдений с условием totwrk=0? 
	Кто эти люди?
	\item Есть ли в датасете пропущенные наблюдения?
	Сколько их?
\end{enumerate}
\end{exercise}

\newpage

\begin{exercise}
Загрузите датасет \texttt{Electricity}.
\begin{enumerate}
	\item вычислите размер датасета (число наблюдений \& число переменных)
	\item заполните следующую таблицу со значениями переменных
	\begin{center}
		\begin{tabular}{|c|c|c|c|c|c|} \hline
			index & cost & q & pl & pk & pf \\ \hline\hline
			1 & & & & & \\ \hline
			15 & & & &  & \\ \hline
			48 & & & & & \\ \hline
			87 & & & & & \\ \hline
		\end{tabular}
	\end{center}
	\item Вычислите корреляционную матрицу для следующих переменных: cost, q, pl, pk, pf 
	\item Заполните следующую таблицу
	\begin{center}
		\begin{tabular}{|c|c|c|c|c|c|} \hline
			Desc.Stat & cost & q & pl & pk & pf\\ \hline\hline
			max & & & & & \\ \hline
			min & & & & & \\ \hline
			mean & & & &  & \\ \hline
			median & & & & & \\ \hline
			st.dev & & & & & \\ \hline
			var (unbiased) & & & & & \\ \hline
			var (biased) & & & & & \\ \hline
			1st quartile & & & & & \\ \hline
			3rd quartile & & & & & \\ \hline
		\end{tabular}
	\end{center}
	Замечание: 1st/3rd квантили -- 25\%/75\% квантили соответственно.
	\item Сколько наблюдения соответствуют следующим условиям
		\begin{enumerate}
			\item cost>40
			\item q<5000
			\item q>4000
			\item 20<cost<50
		\end{enumerate}
	\item Есть ли в датасете пропущенные наблюдения?
	Сколько их?
\end{enumerate}
\end{exercise}

\begin{exercise}
Загрузите датасет \texttt{wage2}.
\begin{enumerate}
	\item вычислите размер датасета (число наблюдений \& число переменных)
	\item заполните следующую таблицу со значениями переменных
	\begin{center}
		\begin{tabular}{|c|c|c|c|c|c|c|} \hline
			index & wage & hours& IQ & educ & exper & age \\ \hline\hline
			1 & & & & & & \\ \hline
			25 & & & &  & & \\ \hline
			179 & & & & & & \\ \hline
			800 & & & & & & \\ \hline
		\end{tabular}
	\end{center}
	\item Вычислите корреляционную матрицу для следующих переменных: wage, hours, IQ, educ, exper 
	\item Заполните следующую таблицу
	\begin{center}
		\begin{tabular}{|c|c|c|c|c|c|c|} \hline
			Desc.Stat & wage & hours& IQ & educ & exper & wage \\ \hline\hline
			max & & & & & & \\ \hline
			min & & & & & & \\ \hline
			mean & & & & & & \\ \hline
			median & & & & & & \\ \hline
			st.dev & & & & & & \\ \hline
			var (unbiased) & & & & & & \\ \hline
			var (biased) & & & & & & \\ \hline
			1st quartile & & & & & & \\ \hline
			3rd quartile & & & & & & \\ \hline
		\end{tabular}
	\end{center}
	Замечание: 1st/3rd квантили -- 25\%/75\% квантили соответственно.
	\item Сколько наблюдения соответствуют следующим условиям
		\begin{enumerate}
			\item wage>1000
			\item age<40
			\item exper>10
			\item 100<IQ<130
		\end{enumerate}
	\item Есть ли в датасете пропущенные наблюдения?
	Сколько их?
\end{enumerate}
\end{exercise}

\begin{exercise}
Загрузите датасет \texttt{Labour}. Создайте новый датасет, 
содержащий log-переменные из исходного датасета.
\end{exercise}

\begin{exercise}
Загрузите датасет \texttt{Electricity}. Создайте новый датасет, 
содержащий log-переменные из исходного датасета.
\end{exercise}

\end{document}