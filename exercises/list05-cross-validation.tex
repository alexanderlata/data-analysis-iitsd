\documentclass[a4,12pt]{article}
\usepackage[utf8]{inputenc}
\usepackage[T2A]{fontenc}
\usepackage[english, russian]{babel}


\usepackage{csquotes}
\usepackage{epigraph}
\usepackage{tcolorbox}
% Определяем окружение для Формата ввода
\newtcolorbox{inputformat}[1][]{colback=blue!5!white, colframe=blue!65!black,
    fonttitle=\bfseries, title=Формат ввода, #1}

% Определяем окружение для Формата вывода
\newtcolorbox{outputformat}[1][]{colback=green!5!white, colframe=green!65!black,
    fonttitle=\bfseries, title=Формат вывода, #1}

% Определяем окружение для комментария
\newtcolorbox{exercisecomment}[1][]{colback=yellow!5!white, colframe=yellow!80!black, fonttitle=\bfseries, title=Комментарий, #1}

% Определяем окружение для замечания
\newtcolorbox{exercisenote}[1][]{colback=red!5!white, colframe=red!80!black, fonttitle=\bfseries, title=Замечание, #1}


\usepackage{geometry}
%\geometry{papersize={20.3 cm,25.4 cm}}
\geometry{left=2cm}
\geometry{right=2cm}
\geometry{top=2cm}
\geometry{bottom=2cm}


\usepackage{amsmath, amsfonts, amsthm, amssymb, amsopn, amscd}
\usepackage{enumerate}
\usepackage{enumitem}
\usepackage[mathscr]{eucal}

\usepackage{hyperref}
\hypersetup{unicode=true,final=true,colorlinks=true}

\theoremstyle{remark}
%\newtheorem{exercise}{Упражнение}
\newtheorem{exercise}{\textbf{Упражнение}}[section]
\renewcommand{\theexercise}{\textbf{\arabic{exercise}}}
%\renewcommand{\theexercise}{\textbf{\#\arabic{exercise}}}



\title{Листок 05. Валидация моделей}

\author{А.Н. Лата}

\begin{document}

\section*{\centering Листок 05. Валидация моделей}

\begin{exercise}
Набор данных \texttt{sleep75} разбейте на обучающую и тестовую часть
в соотношении 80:20.

Рассмотрим задачу прогнозирования для переменных
\begin{center}
	\begin{tabular}{|c|c|}\hline
		зависимая/target & объясняющая/предикторы/features \\ \hline
		sleep & totwrk, age, south, male \\ \hline
	\end{tabular}
\end{center}
и следующие модели
\begin{center}
	\begin{tabular}{|l|l|}\hline
		№ & Модель \\ \hline
		1 & линейная регрессия\\
		2 & k-NN с \(k=5\), веса 'uniform' \\
		3 & k-NN с \(k=5\), веса 'distance' \\
		4 & k-NN с \(k=10\), веса 'uniform' \\
		5 & k-NN с \(k=10\), веса 'distance' \\ \hline
	\end{tabular}
\end{center}
Проведите валидацию моделей относительно метрик \(R^2\), MSE, MAE,
MAPE. Какая модель предпочтительней?
\end{exercise}

\begin{exercise}
Набор данных \texttt{sleep75} разбейте на обучающую и тестовую часть
в соотношении 80:20.

Рассмотрим задачу прогнозирования для переменных
\begin{center}
	\begin{tabular}{|c|c|}\hline
		зависимая/target & объясняющая/предикторы/features \\ \hline
		sleep & totwrk, age, south, male, smsa, yngkid, marr \\ \hline
	\end{tabular}
\end{center}
и следующие модели
\begin{center}
	\begin{tabular}{|l|l|}\hline
		№ & Модель \\ \hline
		1 & линейная регрессия\\
		2 & k-NN с \(k=5\), веса 'uniform' \\
		3 & k-NN с \(k=5\), веса 'distance' \\
		4 & k-NN с \(k=10\), веса 'uniform' \\
		5 & k-NN с \(k=10\), веса 'distance' \\ \hline
	\end{tabular}
\end{center}
Проведите валидацию моделей относительно метрик \(R^2\), MSE, MAE,
MAPE. Какая модель предпочтительней?
\end{exercise}

\begin{exercise}
Набор данных \texttt{wage2} разбейте на обучающую и тестовую часть
в соотношении 80:20.

Рассмотрим задачу прогнозирования для переменных
\begin{center}
	\begin{tabular}{|c|c|}\hline
		зависимая/target & объясняющая/предикторы/features \\ \hline
		wage & age, IQ, south, married, urban \\ \hline
	\end{tabular}
\end{center}
и следующие модели
\begin{center}
	\begin{tabular}{|l|l|}\hline
		№ & Модель \\ \hline
		1 & линейная регрессия\\
		2 & k-NN с \(k=5\), веса 'uniform' \\
		3 & k-NN с \(k=5\), веса 'distance' \\
		4 & k-NN с \(k=10\), веса 'uniform' \\
		5 & k-NN с \(k=10\), веса 'distance' \\ \hline
	\end{tabular}
\end{center}
Проведите валидацию моделей относительно метрик \(R^2\), MSE, MAE,
MAPE. Какая модель предпочтительней?
\end{exercise}

\begin{exercise}
Набор данных \texttt{wage2} разбейте на обучающую и тестовую часть
в соотношении 80:20.

Рассмотрим задачу прогнозирования для переменных
\begin{center}
	\begin{tabular}{|c|c|}\hline
		зависимая/target & объясняющая/предикторы/features \\ \hline
		log(wage) & age, IQ, south, married, urban \\ \hline
	\end{tabular}
\end{center}
и следующие модели
\begin{center}
	\begin{tabular}{|l|l|}\hline
		№ & Модель \\ \hline
		1 & линейная регрессия\\
		2 & k-NN с \(k=5\), веса 'uniform' \\
		3 & k-NN с \(k=5\), веса 'distance' \\
		4 & k-NN с \(k=10\), веса 'uniform' \\
		5 & k-NN с \(k=10\), веса 'distance' \\ \hline
	\end{tabular}
\end{center}
Проведите валидацию моделей относительно метрик \(R^2\), MSE, MAE,
MAPE. Какая модель предпочтительней?
\end{exercise}

\begin{exercise}
Набор данных \texttt{wage1} разбейте на обучающую и тестовую часть
в соотношении 80:20.

Рассмотрим задачу прогнозирования для переменных
\begin{center}
	\begin{tabular}{|c|c|}\hline
		зависимая/target & объясняющая/предикторы/features \\ \hline
		wage & exper, female, married, smsa \\ \hline
	\end{tabular}
\end{center}
и следующие модели
\begin{center}
	\begin{tabular}{|l|l|}\hline
		№ & Модель \\ \hline
		1 & линейная регрессия\\
		2 & k-NN с \(k=5\), веса 'uniform' \\
		3 & k-NN с \(k=5\), веса 'distance' \\
		4 & k-NN с \(k=10\), веса 'uniform' \\
		5 & k-NN с \(k=10\), веса 'distance' \\ \hline
	\end{tabular}
\end{center}
Проведите валидацию моделей относительно метрик \(R^2\), MSE, MAE,
MAPE. Какая модель предпочтительней?
\end{exercise}

\begin{exercise}
Набор данных \texttt{wage1} разбейте на обучающую и тестовую часть
в соотношении 80:20.

Рассмотрим задачу прогнозирования для переменных
\begin{center}
	\begin{tabular}{|c|c|}\hline
		зависимая/target & объясняющая/предикторы/features \\ \hline
		log(wage) & exper, female, married, smsa \\ \hline
	\end{tabular}
\end{center}
и следующие модели
\begin{center}
	\begin{tabular}{|l|l|}\hline
		№ & Модель \\ \hline
		1 & линейная регрессия\\
		2 & k-NN с \(k=5\), веса 'uniform' \\
		3 & k-NN с \(k=5\), веса 'distance' \\
		4 & k-NN с \(k=10\), веса 'uniform' \\
		5 & k-NN с \(k=10\), веса 'distance' \\ \hline
	\end{tabular}
\end{center}
Проведите валидацию моделей относительно метрик \(R^2\), MSE, MAE,
MAPE. Какая модель предпочтительней?
\end{exercise}

\begin{exercise}
Набор данных \texttt{Labour} разбейте на обучающую и тестовую часть
в соотношении 80:20.

Рассмотрим задачу прогнозирования для переменных
\begin{center}
	\begin{tabular}{|c|c|}\hline
		зависимая/target & объясняющая/предикторы/features \\ \hline
		output & capital, labour, wage \\ \hline
	\end{tabular}
\end{center}
и следующие модели
\begin{center}
	\begin{tabular}{|l|l|}\hline
		№ & Модель \\ \hline
		1 & линейная регрессия\\
		2 & k-NN с \(k=5\), веса 'uniform' \\
		3 & k-NN с \(k=5\), веса 'distance' \\
		4 & k-NN с \(k=10\), веса 'uniform' \\
		5 & k-NN с \(k=10\), веса 'distance' \\ \hline
	\end{tabular}
\end{center}
Проведите валидацию моделей относительно метрик \(R^2\), MSE, MAE,
MAPE. Какая модель предпочтительней?
\end{exercise}

\begin{exercise}
Набор данных \texttt{Labour} разбейте на обучающую и тестовую часть
в соотношении 80:20.

Рассмотрим задачу прогнозирования для переменных
\begin{center}
	\begin{tabular}{|c|c|}\hline
		зависимая/target & объясняющая/предикторы/features \\ \hline
		log(output) & log(capital), log(labour), log(wage) \\ \hline
	\end{tabular}
\end{center}
и следующие модели
\begin{center}
	\begin{tabular}{|l|l|}\hline
		№ & Модель \\ \hline
		1 & линейная регрессия\\
		2 & k-NN с \(k=5\), веса 'uniform' \\
		3 & k-NN с \(k=5\), веса 'distance' \\
		4 & k-NN с \(k=10\), веса 'uniform' \\
		5 & k-NN с \(k=10\), веса 'distance' \\ \hline
	\end{tabular}
\end{center}
Проведите валидацию моделей относительно метрик \(R^2\), MSE, MAE,
MAPE. Какая модель предпочтительней?
\end{exercise}

% \section{Валидация моделей и преобразования переменных}

% \begin{exercise}

% \end{exercise}

\end{document}